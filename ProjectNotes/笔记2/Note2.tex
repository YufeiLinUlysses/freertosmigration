\documentclass[a4paper,12pt]{report}

%Packages Used
\usepackage{amssymb,latexsym,amsmath}     % Standard packages
\usepackage{setspace}
\usepackage{sectsty}
\usepackage{titlesec}
\usepackage{hyperref}
\usepackage{bookmark}
\usepackage{graphics,graphicx}
\usepackage{tikz}
\usepackage{mathtools}
\usepackage{graphicx}
\usepackage{esvect}
\usepackage{hyperref}
\usepackage{xcolor}
\usepackage{listings}

\definecolor{mGreen}{rgb}{0,0.6,0}
\definecolor{mGray}{rgb}{0.5,0.5,0.5}
\definecolor{mPurple}{rgb}{0.58,0,0.82}
\definecolor{backgroundColour}{rgb}{0.95,0.95,0.92}

\lstdefinestyle{CStyle}{
    backgroundcolor=\color{backgroundColour},   
    commentstyle=\color{mGreen},
    keywordstyle=\color{magenta},
    numberstyle=\tiny\color{mGray},
    stringstyle=\color{mPurple},
    basicstyle=\footnotesize,
    breakatwhitespace=false,         
    breaklines=true,                 
    captionpos=b,                    
    keepspaces=true,                 
    numbers=left,                    
    numbersep=5pt,                  
    showspaces=false,                
    showstringspaces=false,
    showtabs=false,                  
    tabsize=2,
    language=C
}

\DeclarePairedDelimiter\abs{\lvert}{\rvert}%
\DeclarePairedDelimiter\norm{\lVert}{\rVert}%


\bookmarksetup{
  numbered,
  open
}
\renewcommand*{\thesection}{\arabic{section}}
\onehalfspacing

%Margins
\addtolength{\textwidth}{1.0in}
\addtolength{\textheight}{1.00in}
\addtolength{\evensidemargin}{-0.75in}
\addtolength{\oddsidemargin}{-0.75in}
\addtolength{\topmargin}{-.50in}

%%%%%%%%%%%%%%%%%%%%%%%%%%%%%% 
% Theorem/Proof Environments %
%%%%%%%%%%%%%%%%%%%%%%%%%%%%%%
\newtheorem{theorem}{Theorem}
\newenvironment{proof}{\noindent{\bf Proof:}}{$\hfill \Box$ \vspace{10pt}}
\sectionfont{\fontsize{12}{15}\selectfont}
\titlespacing*{\section}{0.5pt}{0.25\baselineskip}{0.25\baselineskip}

\begin{document}
\noindent
Yufei Lin\\

\noindent
Note\\

\noindent
Jun \(13^{th}\) 2019\\

\noindent
ASTRI Project

\begin{center}
\textbf{Migrate FreeRTOS to Arduino}
\end{center}

\noindent
As of the time I write the document, the newest version of FreeRTOS is 10.2.1.\\

\noindent
\textbf{I. Import Necessary Files}

\noindent
Set up a new folder, include all necessary files in the folder. These files are all files from ``../FreeRTOSv10.2.1/FreeRTOS/Source" folder but in the ``../FreeRTOSv10.2.1/FreeRTOS /Source/portable" folder, we only need the entire folder of ``../GCC/ATMega323". Also, we need to obtain a configuration file from ``../FreeRTOSv10.2.1/FreeRTOS/Demo" folder. It is because using the configuration file from the already existing demo by the author is always easier writing one by oneself. 
Then we obtain the core of FreeRTOS and the port file for FreeRTOS. 

\noindent
Next step, we only need the memory management file. There are five memory management files in the ``../FreeRTOSv10.2.1/FreeRTOS /Source/portable/MemMang" folder. All five of them have different uses. In migrating to Arduino we choose to use ``heap\_3.c". It is because this method provides a protection for all concurrent tasks, which is very necessary for running a concurrent Arduino project. \\

\noindent
\textbf{II. Adjust Files}

\noindent
In order to run this OS on Arduino, we need to write a specific configuration file for Arduino's chip, AVR\_ATMega323. We can name this file as "Arduino\_FreeRTOS.h" Then, we need to replace all FreeRTOS.h configuration file with this newly written configuration file.
 
\end{document}